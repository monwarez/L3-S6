\section{D�finitions et propri�t�s}
	\textit{Probl�me:} Deux droites $D$ et $D'$ s�cantes en un point $M$, et un point $C$. On demande de construire
	la droite passant par $M$ et $C$ sans utiliser $M$. \newline
	\begin{pspicture}(0,0)(5,5)
		\psShapeTwoPtC{black}{-}{0}{1}{5}{1} %% la droite D
		\psShapeTwoPtC{black}{-}{0}{0}{5}{5} %% la droite D'

		\rput(1,1){$\cdot$}
		\rput(1,1.25){$M$} 	%% Le point M
		\rput(5,0){$D$} 	%% Droite D
		\rput(5,4.5){$D'$}	%% Droite D'
		
		%% Le point C
		\rput(4.5,2.5){$\cdot$}
		\rput(5,2.5){$C$}
		%% Soit A sur D'
		\rput(3.5,3.5){$\cdot$}
		\rput(3.5,4){$A$}
		%% Soit B sur D
		\rput(3,1){$\cdot$}
		\rput(3,0){$B$}
		
		%% On trace les droites passant par C et A,B
		\psShapeTwoPtC{green}{-}{4.5}{2.5}{3.5}{3.5} %% la droite AC
		\psShapeTwoPtC{red}{-}{4.5}{2.5}{3}{1} %% la droite BC
		%% Droite de sym�trie AB
		\psShapeTwoPtC{blue}{-}{3}{1}{3.5}{3.5} 
		%% Droite parall�le � BC passant par A
		\psShapeTwoPtC{red}{-}{3.5}{3.5}{0.5}{0.5}	
		%% Droite parall�le � AC passant par B
		\psShapeTwoPtC{green}{-}{3}{1}{1}{3}	
		%% Point C'
		\rput(2,2){$\cdot$}
		\rput(2,2.5){$C'$}
		%% Droite CC'
		\psShapeTwoPtC{yellow}{-}{2}{2}{4.5}{2.5}
		%% la droite MC passe par l'intersection de CC' et AB
		\psShapeTwoPtC{orange}{-}{1}{1}{4.5}{2.5}
 
	\end{pspicture} \newline
	Ok il y a un petit probl�me(j'avais pas eu le temps de copier) \newline
	\begin{definition}
	Soient $X$ un ensemble et $V$ un espace vectoriel sur un corp $\Set{K}$ (ab�lien). \newline
	On dit que $X$ est un espace affine de direction $V$ s'il existe une application de $X \times V \rightarrow X, (x,\vec{v}) \mapsto x + \vec{v}$ \newline
	telle que:
	\begin{enumerate}[(i)]
		\item	$x + \vec{0} = x, \forall x \in X$
		\item	$(x + \vec{u}) + \vec{v} = x + (\vec{u} + \vec{v}) , \forall x \in X, \forall \vec{u},\vec{v} \in V$
		\item	$\forall x \in X, \forall y \in X , \exists ! \vec{v} \in V , y = x + \vec{v}$
	\end{enumerate}
	$\dim_{\Set{K}} X := \dim_{\Set{K}} V$. En plus $\emptyset$ est un espace affine pour lequel $\dim$ n'est pas d�fini.
	\end{definition}
	\begin{remark}
		(i) et (ii) nous montre que $V$ agit sur $X$, (iii) nous montre que l'action est simplement transitive
	\end{remark}
	\begin{definition}
		(Alternative) \newline	
		$X$ est un espace affine dirig� par $V$ s'il existe une action simplement transitive de $V$ sur $X$
	\end{definition}
	\fbox{Notation: } Si $\vec{v} \in V$, on d�finit $\tau_{\vec{v}} : X \rightarrow X, x \mapsto x + \vec{v}$
	

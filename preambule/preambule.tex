%% Preambule.tex	set all extension will used
\usepackage[latin1]{inputenc}
\usepackage[T1]{fontenc}
\usepackage{dsfont}
\usepackage{amssymb}
\usepackage{enumerate}
\usepackage{amsmath}
\usepackage{mathrsfs}
\usepackage[top=2cm, bottom=2cm, left=2cm, right=2cm]{geometry}
\usepackage{soul}
\usepackage{ulem}
\usepackage[Bjornstrup]{fncychap}
\usepackage{fancyhdr}
\usepackage{shorttoc}
\usepackage{tikz}
\usepackage{bookmark}
\usepackage{epsfig}
\usepackage{pst-grad}
\usepackage{pst-plot}
\usepackage[francais]{babel} %% babel last except for listings and hyperref
\usepackage{listings}
\usepackage{hyperref}

\setlength{\headheight}{14.2pt}

\newtheorem{theorem}{Th�or�me}[section]
\newtheorem{lemma}[theorem]{Lemme}
\newtheorem{proposition}[theorem]{Proposition}
\newtheorem{corollary}[theorem]{Corollaire}

\newenvironment{proof}[1][Preuve]{\begin{trivlist}
\item[\hskip \labelsep {\bfseries #1}]}{\end{trivlist}}
\newenvironment{definition}[1][D�finition]{\begin{trivlist}
\item[\hskip \labelsep {\bfseries #1}]}{\end{trivlist}}
\newenvironment{example}[1][Exemple]{\begin{trivlist}
\item[\hskip \labelsep {\bfseries #1}]}{\end{trivlist}}
\newenvironment{remark}[1][Remarque]{\begin{trivlist}
\item[\hskip \labelsep {\bfseries #1}]}{\end{trivlist}}

\newcommand{\qed}{\nobreak \ifvmode \relax \else
      \ifdim\lastskip<1.5em \hskip-\lastskip
      \hskip1.5em plus0em minus0.5em \fi \nobreak
      \vrule height0.75em width0.5em depth0.25em\fi}
%%	for roman numerotation
%%\renewcommand{\thechapter}{\Roman{chapter}}

%%	style front
\fancypagestyle{front}{%
	\fancyhf{}
	\fancyfoot[C]{page \thepage}%
	\renewcommand{\headrulewidth}{0pt}
	\renewcommand{footrulewidth}{0.4pt}
	}
%% style main
\fancypagestyle{main}{
	\fancyhf{}
	\renewcommand{\chaptermark}[1]{\markboth{\chaptername\ \thechapter.\ ##1} {}}
	\renewcommand{\sectionmark}[1]{\markright{\thesection\ ##1}}
	\fancyhead[c]{}
	\fancyhead[RO,LE]{\rightmark}
	\fancyhead[LO,RE]{\leftmark}
	\fancyfoot[C]{}
	\fancyfoot[RO,LE]{page \thepage}
	\fancyfoot[LO,RE]{Alg�bre V}
	}
%% integer segment
\def\segN#1#2{{[\![#1,#2]\!]}}
%% some usefull macro
\def\ord{{\text{ord\,}}}
\def\pgcd#1#2{{\text{pgcd}(#1,#2)\,}}
\def\ppcm#1#2{{ \text{ppcm} (#1,#2)\,}}
\def\Int{{\text{Int\,}}}
\def\Aut{{\text{Auti\,}}}
\def\im{{\text{Im\,}}}
\def\Id{{\text{Id\,}}}
\def\psShapeTwoPt#1#2#3#4#5{{ \psline[linecolor=black, linewidth=0.02, arrowsize=0.052917cm 2.0, arrowlength= 1.4, arrowinset=0.0]{#1}(#2,#3)(#4,#5) }} 
\def\psArrow#1#2#3#4{{ \psShapeTwoPt{->}{#1}{#2}{#3}{#4} }}
\def\diagNoether#1#2#3#4#5#6#7#8{{
	\begin{pspicture}(0,0)(5.5,3.5)
		%we have to do a mapsto
		\psShapeTwoPt{-}{0.9}{2.25}{1.1}{2.25}
  		\psArrow{1}{2.25}{1}{1}
  		\psArrow{2.5}{2.25}{2.5}{1}
  		\psArrow{2.75}{2.5}{4.25}{2.5}
   		\psArrow{2.75}{1}{4.25}{2.25}
               
		\rput(1,2.5){#7}
		\rput(1,0.5){#8}

		\rput(2.5,2.5){#1}
		\rput(2.75,0.5){#2}
		\rput(4.75,2.5){#3}

		\rput(2.25,1.75){#4}
		\rput(4,1.25){#5}
		\rput(3.5,2.75){#6}
   	\end{pspicture}
}}
\def\ProdSemDR#1{{\underset{#1}{\rtimes}}}
\def\Set#1{{\mathds{#1}}}
\def\Mat{{\text{Mat\,}}}
\def\sev{{ \underset{\text{ss-ev}}{\subset}}}

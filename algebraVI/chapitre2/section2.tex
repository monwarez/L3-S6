\section{Orthogonalit� de E* vers E}
	On part d'un sous espace vectoriel $G$ de $E^*$. On va voir $G$ comme un ensemble d'�quations. \newline
	\fbox{Explication} \newline
	"Dual"\qquad \,
	$x^2 + y^2 =1$ , �quation implicite \newline
	"Espace"	$\left\{ \begin{array}{r c l}
	x 	&=&	cos \theta \\
	y 	&=&	sin \theta
	\end{array} \right.$	\newline
	Soit $\varphi \in E^*$ , une �quation dans $E$: $\varphi(X) = 0$. \newline
	Equation implicite: $ax + by +cz = 0$, on a $L = (a,b,c) , L \cdot X = 0$ \newline
	OPS(On peut supposer) $a \ne 0$, on a alors le syst�me suivant:
	\[
		\left\{ \begin{array}{r c l}
		x	&=&	- \frac{b}{a} y - \frac{c}{a} z \\
		y	&=&	y	\\
		z	&=&	z
		\end{array} \right.
	\]
	Alors on a:
	\[
		\begin{pmatrix}
			x	\\
			y	\\
			z	\\
		\end{pmatrix}
		=
		y	
		\begin{pmatrix}
			- \frac{b}{a} 	\\
			\phantom{-} 1 	\\
			\phantom{-} 0
		\end{pmatrix}
		+ z
		\begin{pmatrix}
			- \frac{c}{a} 	\\
			\phantom{-} 0	\\
			\phantom{-} 1	
		\end{pmatrix}
	\]
	Soit $G \subset E^*$
	\begin{definition}
		On pose $G^o \subset E$ l'ensemble
		\[
			G^o = \{ x \in E, \varphi (x) = 0, \forall \varphi \in G \}
		\]
		$G^o$ est appel� l'orthogonal de $G$, on v�rifie en exercice que $G^o$ est bien un sous espace vectoriel de $E$
	\end{definition}
	\begin{proposition}
		Soient $G,G_1,G_2 \sev E^*$ , alors on a les propri�t� suivantes: \newline
		\begin{enumerate}
			\item	$\dim G^o + \dim G = \dim E$
			\item	$G_1 \subset G_2 \Leftrightarrow G_1^o \supset G_2^o$
			\item	$(G_1 + G_2)^o 	= G_1^o \cap G_2^o$ 
			\item	$(G_1 \cap G_2)^o	=	G_1^o + G_2^o$
		\end{enumerate}
	\end{proposition}	 
	\begin{proof}
		\begin{enumerate}
			\item	On reconnait la formule du rang. $G$ est un sous espace vectoriel, donc poss�de une base $(\varphi_i)_{ 1 \le i \le d}$ avec 
					$d	=	\dim G$, on a:
					\[
						v \in G^o \Leftrightarrow \forall \varphi \in G , \varphi(v) = 0 \Leftrightarrow \forall i, 1 \le i \le d, \varphi_i(v) = 0
					\]
					Soit $X = \Mat_\mathscr{B} (x) , L_i = \Mat_\mathscr{B^*} (\varphi_i) $
					\[
						\forall i , \varphi_i(x) = 0 \Leftrightarrow L_i X = 0, \forall i \Leftrightarrow
						\begin{pmatrix}
							L_1 \\
							\vdots \\
							L_n
						\end{pmatrix}
						X	=	0
					\]
					On pose $L = 
					\begin{pmatrix}
						L_1 \\
						\vdots \\
						L_n
					\end{pmatrix}
					\in \Mat_{n,n}(\Set{K}) $, alors on a:
					\[
						\dim G^o = \dim \{ X, LX = 0 \} = \dim \ker L = \dim E - \dim \im L = \dim E - \dim <L_i>_{1 \le i \le d} = \dim E - \dim G
					\]
	
		\end{enumerate}
	\end{proof}
